% Enable hyperlinks
\setupinteraction
  [state=start,
  style=,
  color=blue,
  contrastcolor=]

% make chapter, section bookmarks visible when opening document
\placebookmarks[chapter, section, subsection, subsubsection, subsubsubsection, subsubsubsubsection][chapter, section]
\setupinteractionscreen[option=bookmark]
\setuptagging[state=start]

\setuplayout[rightmargin=1in,leftmargin=1in,bottom=1in,top=1in,header=0mm]
\setuppagenumbering[location=footer]

% use microtypography
\definefontfeature[default][default][script=latn, protrusion=quality, expansion=quality, itlc=yes, textitalics=yes, onum=yes, pnum=yes]
\definefontfeature[smallcaps][script=latn, protrusion=quality, expansion=quality, smcp=yes, onum=yes, pnum=yes]
\setupalign[hz,hanging]
\setupitaliccorrection[global, always]

\setupbodyfontenvironment[default][em=italic] % use italic as em, not slanted

\definefallbackfamily[mainface][rm][CMU Serif][preset=range:greek, force=yes]
\definefontfamily[mainface][rm][Latin Modern Roman]
\definefontfamily[mainface][mm][Latin Modern Math]
\definefontfamily[mainface][ss][Latin Modern Sans]
\definefontfamily[mainface][tt][Latin Modern Typewriter][features=none]
\setupbodyfont[mainface]

\setupwhitespace[medium]

\setuphead[chapter]            [style=\tfd,header=empty]
\setuphead[section]            [style=\tfc]
\setuphead[subsection]         [style=\tfb]
\setuphead[subsubsection]      [style=\bf]
\setuphead[subsubsubsection]   [style=\sc]
\setuphead[subsubsubsubsection][style=\it]

\setuphead[chapter, section, subsection, subsubsection, subsubsubsection, subsubsubsubsection][number=no]

\definedescription
  [description]
  [headstyle=bold, style=normal, location=hanging, width=broad, margin=1cm, alternative=hanging]

\setupitemize[autointro]    % prevent orphan list intro
\setupitemize[indentnext=no]

\setupfloat[figure][default={here,nonumber}]
\setupfloat[table][default={here,nonumber}]

\setupthinrules[width=15em] % width of horizontal rules

\setupxtable[frame=off]
\setupxtable[head][topframe=on,bottomframe=on]
\setupxtable[body][]
\setupxtable[foot][bottomframe=on]


\starttext

\section[title={Writeup for CTF challenge}]

Challenge: Malaisia Student name: Nam Xuan Nguyen Student ID:

\section[title={Requirement}]

A picture and a Python code are provided.
{\externalfigure[./tex2pdf.-4482776c2dfe93ab/98c6912f1ceb420e0b0690ffd0fa1b9ab61d287e.png]}

\starttyping
from PIL import Image
import random
import sys
def qq(x, y):
    return (2 * x + 3 * y + 29) % 256
def transform(pixelinfo):
    pixelreverse = [pixelinfo[len(pixelinfo)-1-i] for i in range(len(pixelinfo))]
    out = [pixelinfo[i] for i in range(len(pixelinfo))]
    for i in range(len(pixelinfo)):
        out[0] = qq(pixelreverse[i], out[0])
        for j in range(1,len(pixelinfo)):
            out[j] = qq(out[j-1], out[j])
    return out
image = Image.open(sys.argv[1])
outfile1 = Image.new(image.mode, image.size)
for x in range(0, image.size[0]):
    for y in range(0, image.size[1]):
        sourcepixel = list(image.getpixel((x, y)))
        tran = transform(sourcepixel)
        outfile1.putpixel((x, y), tuple(tran))
outfile1.save('out1.bmp')
\stoptyping

It is easy to see that this picture is the result of an tranformation
using the script. So our target is to reverse this transformation to
find the original picture.

\section[title={Solution}]

We need to build an inverse function of the above transformation. . The
format of this picture is bmp, that means each pixel includes 3 values
\type{(R, G, B)} in which {\em R}, {\em G}, {\em B} go from {\em 0} to
{\em 255}. Therefore, if we build an 3-dimension array and each member
has the size of 1 byte, so it requires \type{256^3 * 256^3 byte = 64MB}.
Each item of this array at position {\em {[}a, b, c{]}} is the origin
pixel of pixel \type{(a,b,c)}. As the required memory is quite small, it
is posible to use this solution.

This is the {\em Python} code that I used to reverse the picture

\starttyping
from PIL import Image

import random
import sys

def qq(x, y):
    return (2 * x + 3 * y + 29) % 256

def transform(pixelinfo):
    pixelreverse = [pixelinfo[len(pixelinfo)-1-i] for i in range(len(pixelinfo))]
    out = [pixelinfo[i] for i in range(len(pixelinfo))]
    for i in range(len(pixelinfo)):
        out[0] = qq(pixelreverse[i], out[0])
        for j in range(1,len(pixelinfo)):
            out[j] = qq(out[j-1], out[j])
    return out


image = Image.open(sys.argv[1])
outfile1 = Image.new(image.mode, image.size)
# My code start from here
listReverse=[[]]*256*256*256
hundred=256*256
ten=256

for x in range(0, 256):
    for y in range(0, 256):
        for z in range(0, 256):
            sourcepixel = [x,y,z]
            tran = transform(sourcepixel)
            listReverse[tran[0]*hundred+tran[1]*ten+tran[2]]=sourcepixel

for x in range(0, image.size[0]):
    for y in range(0, image.size[1]):
        sourcepixel = list(image.getpixel((x, y)))
        tran = listReverse[sourcepixel[0]*hundred + sourcepixel[1]*ten+sourcepixel[2]]
        outfile1.putpixel((x, y), tuple(tran))


outfile1.save('out1.bmp')
\stoptyping

The first step is to prepare the array. I ran 3 nested \type{for} loops
with indexes \type{x, y, z}. With each pixel \type{(x, y, z}), I found
the its trasformation pixel,\type{(tR, tG, tB)} then assigning
\type{(x, y, z)} to an item of the array at position (tR, tG, tB)
because (x, y, z) is converted to (tR, tG, tB).

After finishing the array, I ran two nested \type{for} loops to go
through all positions in the picture. At the position (x, y), I got the
pixel value, looked up on the array to get the original pixel, and put
it back to that position in the output picture.

The output picture is the original one, containing the flag.

In my code, I avoided to use 3-dimension array. Insteadly, I used a
1-dimension array having \type{256^3} items. I converted the index (a,
b, c) to \type{256^2 * a + 256 * b + c = d} and used \type{d} as the
index in my array.

The flag is in this picture
{\externalfigure[./tex2pdf.-4482776c2dfe93ab/4b566f87daa0d128473f87dcffc5f63190245923.png]}

\stoptext
